\documentclass{article}

\title{Using \LaTeX~in category theory}
\author{Egbert Rijke\\\texttt{erijke@andrew.cmu.edu}}
\date\today

%%%% Packages
\usepackage{microtype}              % adjusts LaTeX's default spacing
\usepackage{amsmath,amssymb,amsthm} % display math
\usepackage{tikz-cd}                % draw commutative diagrams
\usepackage{listings}               % display code
\lstset{%                            % set parameters to display code
  language=tex,%
  keywordstyle=\color{red}\texttt,%
  commentstyle=\color{blue}\textit,%
  numbers=left,%
  numberstyle=\tiny\color{gray}%
}
\lstset{%
  morekeywords={documentclass,begin,author,title,date,today,maketitle,LaTeX,
  newcommand,mathcal,mathbf,mathrm,texttt,theoremstyle,newtheorem,emph,circ}%
}
\usepackage[colorlinks=true]{hyperref}               % create hyperlinks in the pdf file.

%%%% Environments for theorems and definitions
\theoremstyle{definition}
\newtheorem{defn}{Definition}[section]
\newtheorem{ex}{Exercise}

%%%% Macros for mathematical notation
\newcommand{\cat}{%
  \mathbf%
}
\newcommand{\domain}[1]{%
  \mathrm{dom}(#1)%
}
\newcommand{\codomain}[1]{%
  \mathrm{cod}(#1)%
}
\newcommand{\idarrow}[1][]{%
  \mathbf{1}_{#1}%
}

%%%% Macros for use in text
\newcommand{\code}{%
  \texttt%
}
\newcommand{\name}{%
  \textsc%
}
  
%%%% Environments

%%%% Set custom controls for enumerate environment
\renewcommand{\theenumi}{\alph{enumi}}
\renewcommand{\labelenumi}{(\theenumi)}
\renewcommand{\theenumii}{\roman{enumii}}
\renewcommand{\labelenumii}{(\theenumii)}



\begin{document}

\maketitle

This document is a quick how-to for the students of the category theory
class at the CMU. It contains specific and practical knowledge of \LaTeX~which
will be useful to write homework solutions.

\section{Writing your document}
\subsection{Setting up your document}

To get started with \LaTeX, section 1 of the \name{Wikibook} on
\url{http://en.wikibooks.org/wiki/LaTeX} provides descriptions of what \LaTeX~is,
how to obtain it and the basics of \LaTeX-syntax.

The basic layout of the source code of the document you are currently
reading is

\begin{quote}
\begin{minted}{tex}
\documentclass{article}

\title{Using \LaTeX~in category theory}
\author{Egbert Rijke}
\date\today

% Before you begin writing your document you have
% space to set up user-defined commands, or to 
% load existing packages containing frequently
% used commands. This part of the file is called
% the preamble.

\begin{document}
\maketitle

% Write the text you want the document to contain
% between the \begin{document} and the 
% \end{document} declarations.

\end{document}
\end{minted}
\end{quote}

As soon as the \LaTeX-compiler encounters the percent sign (\%), the rest of 
that line including the line-break, is ignored by the compiler. You can use this 
feature to provide comments within your source code, or to keep your code neat
and organized.

The document structure is explained in more detail on
\url{http://en.wikibooks.org/wiki/LaTeX/Document_Structure}

\subsection{User defined macros}

User defined commands, also called \emph{macros}, are useful to
\begin{itemize}
\item simplify the process of typesetting,
\item make the source code of your document (i.e.~the \code{.tex} files) more readable,
\item ensure uniformity of notation in your document, or even throughout several
documents,
\item facilitate change of notation when you're not yet quite sure.
\end{itemize}
As a rule of thumb, every mathematical concept which needs notation has
either a predefined macro (in the case it is very basic), or it has a
user defined macro.

The macros defined to write the math in this document are
\begin{quote}
\begin{minted}{tex}
\newcommand{\cat}{%
  \mathbf%
}
\newcommand{\domain}[1]{%
  \mathrm{dom}(#1)%
}
\newcommand{\codomain}[1]{%
  \mathrm{cod}(#1)%
}
\newcommand{\idarrow}[1][]{%
  \mathbf{1}_{#1}%
}
\end{minted}
\end{quote}

Detailed information on defining your own macros can be found on
\url{http://en.wikibooks.org/wiki/LaTeX/Macros}

\subsection{Using the same preamble for several documents}
It is a good idea to have a separate file \code{preamble.tex}
for the preamble and include it just above \verb+\begin{document}+ with the line
\begin{quote}
\begin{minted}{tex}
%%%% Packages
\usepackage{microtype}              % adjusts LaTeX's default spacing
\usepackage{amsmath,amssymb,amsthm} % display math
\usepackage{tikz-cd}                % draw commutative diagrams
\usepackage{listings}               % display code
\lstset{%                            % set parameters to display code
  language=tex,%
  keywordstyle=\color{red}\texttt,%
  commentstyle=\color{blue}\textit,%
  numbers=left,%
  numberstyle=\tiny\color{gray}%
}
\lstset{%
  morekeywords={documentclass,begin,author,title,date,today,maketitle,LaTeX,
  newcommand,mathcal,mathbf,mathrm,texttt,theoremstyle,newtheorem,emph,circ}%
}
\usepackage[colorlinks=true]{hyperref}               % create hyperlinks in the pdf file.

%%%% Environments for theorems and definitions
\theoremstyle{definition}
\newtheorem{defn}{Definition}[section]
\newtheorem{ex}{Exercise}

%%%% Macros for mathematical notation
\newcommand{\cat}{%
  \mathbf%
}
\newcommand{\domain}[1]{%
  \mathrm{dom}(#1)%
}
\newcommand{\codomain}[1]{%
  \mathrm{cod}(#1)%
}
\newcommand{\idarrow}[1][]{%
  \mathbf{1}_{#1}%
}

%%%% Macros for use in text
\newcommand{\code}{%
  \texttt%
}
\newcommand{\name}{%
  \textsc%
}
  
%%%% Environments

%%%% Set custom controls for enumerate environment
\renewcommand{\theenumi}{\alph{enumi}}
\renewcommand{\labelenumi}{(\theenumi)}
\renewcommand{\theenumii}{\roman{enumii}}
\renewcommand{\labelenumii}{(\theenumii)}


\end{minted}
\end{quote}
The \verb+\input+ command looks for the \code{.tex} file with the name provided
in the braces in the current folder, unless you give it specific instructions to
look elsewhere. Thus, the file \code{preamble.tex} must in this case be in the
same folder as the main file.

Using a seperate preamble file allows you to use the same preamble for several
documents. This is actually a first step in creating your own packages, see
\url{http://en.wikibooks.org/wiki/LaTeX/Creating_Packages}.

\section{Category theory in \LaTeX}

\subsection{Theorems, definitions and exercises}
Theorem environments, which are provided by the \code{amsthm} package, can be
used to declare environments for theorems, lemmas, definitions, exercises, and
the like. To declare the environments for definitions and exercises, we have
included the lines
\begin{quote}
\begin{minted}{tex}
\theoremstyle{definition}
\newtheorem{defn}{Definition}[section]
\newtheorem{ex}{Exercise}
\end{minted}
\end{quote}
in the preamble.

Now we can start writing definitions. For instance, the definition of a category
is written in the \code{document}-environment as:

\begin{quote}
\begin{minted}{tex}
\begin{defn}
A \emph{category} \(\cat{C}\) consists of
\begin{itemize}
\item a collection of objects: \(A\), \(B\), \(C\),
  \ldots
\item a collection of arrows: \(f\), \(g\), \(h\),
  \ldots
\item for each arrow \(f\) objects \(\domain{f}\) and
  \(\codomain{f}\) called the \emph{domain} and 
\emph{codomain} of \(f\). If \(\domain{f}=A\) and 
  \(\codomain{f}=B\), we also write \(f:A\to B\),
\item given \(f:A\to B\) and \(g:B\to C\), so that 
  \(\domain{g}=\codomain{f}\), there is an arrow 
  \(g\circ f:A\to C\),
\item an arrow \(\idarrow[A]:A\to A\) for every 
  object \(A\) of \(\cat{C}\),
\end{itemize}
such that
\begin{description}
\item[(Associative law)] for every \(f:A\to B\), 
  \(g:B\to C\) and \(h:C\to C\) we have
\begin{equation*}
h\circ(g\circ f)=(h\circ g)\circ f,
\end{equation*}
\item[(Unit laws)] for every \(f:A\to B\) we have
\begin{equation*}
f\circ\idarrow[A]=f=\idarrow[B]\circ f.
\end{equation*}
\end{description}
\end{defn}
\end{minted}
\end{quote}

This results in:
\begin{defn}
A \emph{category} \(\cat{C}\) consists of
\begin{itemize}
\item a collection of objects: \(A\), \(B\), \(C\), \ldots
\item a collection of arrows: \(f\), \(g\), \(h\), \ldots
\item for each arrow \(f\) objects \(\domain{f}\) and \(\codomain{f}\) called
the \emph{domain} and \emph{codomain} of \(f\). If \(\domain{f}=A\) and \(\codomain{f}=B\),
we also write \(f:A\to B\),
\item given \(f:A\to B\) and \(g:B\to C\), so that \(\domain{g}=\codomain{f}\), there
is an arrow \(g\circ f:A\to C\),
\item an arrow \(\idarrow[A]:A\to A\) for every object \(A\) of \(\cat{C}\),
\end{itemize}
such that
\begin{description}
\item[(Associative law)] for every \(f:A\to B\), \(g:B\to C\) and \(g:C\to D\) we have
\begin{equation*}
h\circ(g\circ f)=(h\circ g)\circ f,
\end{equation*}
\item[(Unit laws)] for every \(f:A\to B\) we have
\begin{equation*}
f\circ\idarrow[A]=f=\idarrow[B]\circ f.
\end{equation*}
\end{description}
\end{defn}

\subsection{Drawing diagrams}
For drawing diagrams, we recommend the \code{tikz-cd} package. The documentation
for the \code{tikz-cd} package is available on \url{http://www.ctan.org/pkg/tikz-cd}.
Be sure you use the latest version of \code{tikz-cd}, because its features and
syntax has been changed recently.

As an example of a diagram drawn with \code{tikz-cd}, the following code displays
the diagram that has been used in class to demonstrate the associative law:
\begin{quote}
\begin{minted}{tex}
\begin{equation*}
\begin{tikzcd}
A \arrow[r,"f"]
  \arrow[dr,swap,"g\circ f"]
  &
B \arrow[dr,"g\circ h"]
  \arrow[d,swap,"g"]
  \\
  {}&
C \arrow[r,swap,"h"]
  &
D
\end{tikzcd}
\end{equation*}
\end{minted}
\end{quote}
The output of the above code is the diagram
\begin{equation*}
\begin{tikzcd}
A \arrow[r,"f"]
  \arrow[dr,swap,"g\circ f"]
  &
B \arrow[dr,"g\circ h"]
  \arrow[d,swap,"g"]
  \\
  {}&
C \arrow[r,swap,"h"]
  &
D
\end{tikzcd}
\end{equation*}
It is also possible to draw parallel arrows, for instance, to display coequalizer
diagrams, and dotted arrows to display universal properties. The code
\begin{quote}
\begin{minted}{tex}
\begin{equation*}
\begin{tikzcd}
A \arrow[yshift=.7ex,r,"f"]
  \arrow[yshift=-.7ex,r,swap,"g"]
  &
B \arrow[r,"e"]
  \arrow[dr,swap,"h"]
  &
C \arrow[densely dotted,d,"\exists!"]
  \\
& &
D
\end{tikzcd}
\end{equation*}
\end{minted}
\end{quote}
results in the diagram
\begin{equation*}
\begin{tikzcd}
A \arrow[yshift=.7ex,r,"f"]
  \arrow[yshift=-.7ex,r,swap,"g"]
  &
B \arrow[r,"e"]
  \arrow[dr,swap,"h"]
  &
C \arrow[densely dotted,d,"\exists!"]
  \\
& &
D
\end{tikzcd}
\end{equation*}
When the diagram seems too packed with information, it sometimes helps to
separate the rows and columns more by starting the \code{tikzcd}-environment
with the option \code{column sep=huge} or \code{row sep=large}, for example.
More precisely, the environment would look like
\begin{quote}
\begin{minted}{tex}
\begin{tikzcd}[column sep=huge]
...
\end{tikzcd}
\end{minted}
\end{quote}
The package documentation contains more examples which will prove useful in
displaying diagrams.

\section{Getting your homework done}
The source file of your homework would probably look something like

\begin{quote}
\begin{minted}{tex}
\documentclass{article}
\title{Homework set 1}
\author{Student's name}
\date\today

\usepackage{amssymb,amsthm,amsmath}
\usepackage{tikzcd}

%%%% The exercise environments
\theoremstyle{definition}
\newtheorem{ex}{Exercise}

%%%% My macros
\newcommand{\cat}{%
  \mathbf%
}
\newcommand{\domain}[1]{%
  \mathrm{dom}(#1)%
}
\newcommand{\codomain}[1]{%
  \mathrm{cod}(#1)%
}
\newcommand{\idarrow}[1][]{%
  \mathbf{1}_{#1}%
}

%%%% Macros for specific categories
\newcommand{\Cat}{%
  \cat{Cat}%
}
\newcommand{\Mon}{%
  \cat{Mon}%
}
\newcommand{\Poset}{%
  \cat{Poset}%
}
\newcommand{\Rel}{%
  \cat{Rel}%
}
\newcommand{\Sets}{%
  \cat{Sets}%
}
\newcommand{\Groups}{%
  \cat{Groups}%
}
\newcommand{\Graphs}{%
  \cat{Graphs}%
}

\begin{document}
\maketitle

\begin{ex}
% Solution to exercise 1
\end{ex}

\begin{ex}
% Solution to exercise 2
\end{ex}

\begin{ex}
\begin{enumerate}
\item % Solution to exercise 3a
\item % Solution to exercise 3b
\item % Solution to exercise 3c
\end{enumerate}
\end{ex}

\end{document}
\end{minted}
\end{quote}
You are free to copy and use the above code.

\section{More \LaTeX-resources}

\begin{itemize}
\item Books:
\begin{itemize}
\item The \LaTeX~companion (2nd edition) by Mittelbach, Goossens and Braams,
\item The \TeX-book by Knuth,
\item More math into \LaTeX~by Gr\"atzer.
\end{itemize}
\item Resources on the web:
\begin{itemize}
\item The main question \& answer site with lots of useful tips and tricks
is \url{http://tex.stackexchange.com}.
\item Whenever you want to use a symbol in your document, but don't know the
corresponding \LaTeX-command, or even how to call that symbol, \name{detexify}
might be an outcome. On the website \url{http://detexify.kirelabs.org/classify.html}
you can draw your symbol and it will tell you which of the actual
\LaTeX-symbols it thinks it matches most closely. If this does not help, consult
the \name{ctan} comprehensive symbol list, which is available at
\url{http://www.ctan.org/tex-archive/info/symbols/comprehensive/}
\end{itemize}
\end{itemize}
\end{document}
